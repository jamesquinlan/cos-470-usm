%+++++++++++++++++++++++++++++++++++++++++++++++++++++++++++++++
% SUMMARY    : Extra Scribe 1 (Masters Student Assignment)
%            : University of Southern Maine 
%            : @james.quinlan
%            : Sarah Lawrence 
%+++++++++++++++++++++++++++++++++++++++++++++++++++++++++++++++


\section*{Objectives}
\begin{itemize}
    \item Objective 1
    \item Objective 2
    \item Objective 3
\end{itemize}

\rule[0.0051in]{\textwidth}{0.00025in}
% ----------------------------------------------------------------


\section{Theorem}
if ${u_1,u_2,...,u_u}$ Orthonormal and $v=\sum c_iu_i$ then $c_i=<u_i,v>$\\
\subsection{Proof}
\begin{align*}
<u_i,v> &=<u_i,\sum_{j=1}^{n}c_ju_j\\
&=\sum c_j <u_{ji}u_i>\\
&=\sum_{j=1}^{n} c_jf_{ij}\\
&= c_i
\end{align*}
\textbf{Based on:} $<ax+y,z>=a<x,z>+<y,z>$ \\
\textbf{Corollary:} if $u=\sum a_iu_i \land \lor =\sum b_iu_i$
\subsection{Proof}
\begin{align*}
<u,v> &=<a_iu_i,b_iu_i>\\
&=a_1<u_1,\sum b_iu_i>+a_2<u_2,v>+...\\
&=a_1b_1<u_1,u_1>+a_1b_2<u_1,u_2>+...\\
&+a_2b1<u_2,u_1>+...\\
<u,v> &=<\sum a_i u_i,v>\\
&=\sum a_i<u_i,v>\\
&=\sum a_i <v_1u_i>\\
&=\sum a_i b_i\\
\end{align*}

\subsection{Definitions}
\subsubsection{Def 1}
$||v||=\sqrt{<v,v>}=\sqrt{v^tv}$
\subsubsection{Def 2}
$O\in \mathbb{R}^n*n$ is or orthogonal if $Qi$ from orthogonal set.
\subsubsection{Def 3}
Orthogonal set $\{u_1,u_2,...,u_u\}$ \\
Orthogonal unit vectors\\
\[
u_i^tu_j=\delta_{ij}= 
\begin{cases}
1 & \text{if } i = j \\
0 & \text{if } i \ne j
\end{cases}
\]
\subsubsection{Def 4}
Gwen avg $\{v_1,v_2,...,v_u\}$ orthogonal\\
$u_i = \frac{v_i}{||v_i||}$ are normalized, hence orthogonal\\
\textbf{Ex:}
\begin{align*}
v_1 &= (1, 1, 1)^\top \rightarrow\quad u_1 = \frac{1}{\sqrt{3}} v_1 \\
v_2 &= (2, 1, -3)^\top \rightarrow\quad u_2 = \frac{1}{\sqrt{14}} v_2 \\
v_3 &= (4, -5, 1)^\top \rightarrow\quad u_3 = \frac{1}{\sqrt{42}} v_3
\end{align*}
\subsubsection{Def 5}
$a|b$ iff $\exists k \in \mathbb{Z}: b=ak$\\
\begin{itemize}
  \item b is a multiple of a
  \item a is a divisor of b
  \item a is a factor of b
\end{itemize}
\subsubsection{Def 6}
$n>1$ is prime is only divisors a $1 \land n$. (no factors)
\subsubsection{Def 7}
Composite is not prime (has at least 1 factor).
\subsubsection{Def 8}
If a/n $\land a/m$, then a is a common factor. 
\subsubsection{Def 9}
gcd$(m,n)=d \Leftrightarrow d|m \land d|u$ and if $h|m \land h|n$, then $d \leq h$

\section{Theorem}
if $h|n \land h|m$, then $\forall i,j  h|(in+jm)$. 

\subsection{Proof}
$n=nk$, $\land m=hk_2$ then 
\begin{align*}
in+jm &=ihk_1+jhk_2\\
&=h(ik_1+jk_2)\\
\rightarrow & h|(in+jm)
\end{align*}

\subsection{Definition}
(Quotient) $\forall m,n:m\ne0$ 
$q= \lfloor \frac{n}{m} \rfloor$ and (remainder) of dividing n by m is $r=n-qm$. ($r=n$ mod $m)$ \\
i.e. 
\[
\boxed{n = mq + r \quad \text{such that} \quad
\begin{cases}
0 & \text{if} \ 0 \leq r < m  \quad m > 0\\
m & \text{if} \ m \leq r \leq 0  \quad m < 0
\end{cases}
}
\]

\section{Theorem}
let $m,n \in \mathbb{Z}: m \ne 0 V n\ne0 $ and (not both 0) let $d=min\{in+jm$such that$i,j\in \mathbb{Z} \land in+jm >0\}$ then $d=gcd(m,n)$. 

\subsection{Proof}
$d=in+jm$.(smallest linear combo of $m \land n$, moreover, $n=qd+r$ with  $0 \leq r < d$. Therefor we have 
\begin{align*}
n&=q(in+jm)+r\\
&=qin+qjm+r\\
r&=n(1-qi)+m(qi)
\end{align*}
Since r is linear combo but d is minimal $\land r<d$, then $r=0$, $\rightarrow d|n $. Sumilarly $d|m$ $\rightarrow d$ is common divisor $\rightarrow d \leq gcd(m,n)$. Also since $gcd(m,n)$ divides $m \land n$ and $d=in+jm$, then $gcd(m,n)|d \rightarrow d=gcd(m,n)k \rightarrow d \geq acd(m,n)$
\textbf{Corollary:} if $h|n \land h|m$, then $h|gcd(m,n)$ 

\subsection{Proof}
$gcd(m,n) = am+bn$\\
$\therefore$ since $h|m \land h|n$ then $m=hk_1 \land n=hk_2$ then
\begin{align*}
gcd(m,n)&= hak_1 + hbk_2\\
&=h(ak_1+dk_2)\\
\end{align*}

\section{Theorem}
Suppose $n \geq 0 \quad m>0 $. Let $r=n \quad mod m$ then $gcd(m,n)=gcd(m,r).$ 
\subsection{Proof}
show $gcd(m,n)|gcd(m,n)$ and vice versa. Let $d_1 = gcd(m,n)$, then $d_1|m \land d_1/n \rightarrow m=d_1k_1 \land n=d_1k_2$. Also $n=mq+r \quad \exists q_1r$:$0\leq r <m$

\section{Gradient Descent}
\begin{itemize}
  \item Compute sequence of vectors in $\mathbb{R}^s$ aiming to converge to a vector that minimizes cost
  \item if $\Delta p \approx0$,then Taylor Series gives $Cost(p+\Delta p) \approx Cost(p)+ \sum_{r=1}^{s} \frac{\partial Cost(p)}{\partial pr}$.$\Delta pr$. $Cost(p)+ \frac{\partial C(p)}{\partial p} +\frac{\partial C(p)}{\partial p_2}+...+\frac{\partial C(p)}{\partial p_2}$
  \item Denote $\nabla C(p) \in \mathbb{R}^s$, gradient (vector of partial derivatives). 
  \item $\frac{\partial C(p)}{\partial pi} =$ how much Cost changes in the "direction" pi.
  \item $(\nabla C(p))_r = \frac{\partial C(p)}{\partial p_r}$
  \item $Cost (p+\Delta p) \approx Cost(p)+ \nabla Cost(p)^t\Delta p $. \\
\[
\boxed{Cost (p + \Delta p) \approx C(p) + \nabla C(p)^t \Delta p}
\]
  \item Choose $\Delta p $ to make $ \nabla C(p)^t\Delta p$ as negative as possible.
  \item Cauchy - Schwarz Inequality. \\
  $|x^ty| \leq||x||_2*||y||_2$\\
  $\therefore -||x||_2*||y||_2  \leq x^ty \leq||x||_2*||y||_2$\\
  $-||x||_2*||y||_2$ most negtive $x^ty$ can be.\\
  $\rightarrow$ When $x=-y$.
\end{itemize}
\textbf{Note:} $|x^y|=|x*y|=||x||_2*||y||_2*cos0 \quad x= \lambda y$, then $|\lambda y*y|=|\lambda|*||y||^2=||y||*||y||*|cos0|$\\
$||x||=\sqrt{x^tx}$\\
$||y|| = $
